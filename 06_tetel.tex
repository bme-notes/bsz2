\documentclass[]{article}
\usepackage{lmodern}
\usepackage{amssymb}
\usepackage{amsmath}
\usepackage{polyglossia}
\usepackage{listings}
\usepackage{tcolorbox}
\usepackage{etoolbox}
\usepackage{setspace}
\usepackage{framed}
\usepackage[a4paper,margin=2cm,footskip=.5cm]{geometry}
\newcommand{\R}{\mathbb{R}}
\newcommand{\Rn}[1]{$\mathbb{R}^{#1}$}
\newcommand{\Und}[1]{\underline{#1}}
\definecolor{shadecolor}{gray}{0.9}
%opening 
\title{Bevezetés a Számításelméletbe 2.\\{\large 6. tétel}}
\author{Hegyi Zsolt}
\begin{document}
\maketitle
\begin{shaded}
EULER-ÚT ÉS KÖR Definíció: A G gráf \textbf{Euler-körének} nevezzük egy zárt élsorozatot, ha az élsorozat pontosan egyszer tartalmazza G összes élét. Ha az élsorozat nem feltétlenül zárt, akkor \textbf{Euler-útról} beszélünk.
\end{shaded}
\begin{framed}
EULER-KÖR LÉTEZÉSE Tétel: G összefüggő gráfban akkor és csak akkor létezik Euler-kör, ha G minden pontjának fokszáma páros.
\end{framed}
\begin{leftbar}
Először lássuk be, hogy ha van a gráfban E-kör, akkor minden pont foka páros. Induljunk el a gráf tetszőleges pontjából és járjuk körbe az E-kör mentén. Minden pontban ugyanannyiszor mentünk be, mint ahányszor kimentünk, a ki-bemenések száma a pont fokszáma. Ez biztosan páros. Másik irány jegyzet 29. oldal, baszódjon meg ez a bizonyítás.
\end{leftbar}
\begin{framed}
EULER-ÚT LÉTEZÉSE Tétel: Egy összefüggő G gráfban akkor és csak akkor létezik Euler-út, ha a páratlan fokszámú pontok száma 0 vagy 2.
\end{framed}
\begin{leftbar}
Az előző tétel bizonyítása alapján, ebben az esetben ha 0 a páratlan fokszámú pontok száma, akkor Euler-körről is beszélhetünk, ha 2, akkor az élsorozat nem zárt, a két végpontnak lesz eltérő a fokszáma, mivel ezt úgy tudjuk képezni, hogy a két végpontot összekötő élt elhagyjuk.
\end{leftbar}
\begin{shaded}
HAMILTON-ÚT ÉS KÖR Definíció: Egy G gráfban Hamilton-körnek nevezünk egy H kört, ha G minden pontját pontosan egyszer tartalmazza. Egy utat pedig Hamilton-útnak nevezünk, ha G minden pontját pontosan egyszer tartalmazza.
\end{shaded}
\begin{framed}
SZÜKSÉGES FELTÉTEL A H-KÖR (ÚT) LÉTEZÉSÉHEZ Tétel: Ha a G gráfban létezik k olyan pont, melyeket elhagyva a gráf több mint k komponensre esik, akkor nem létezik a gráfban H-kör. Ha több, mint k+1 komponensre esik, akkor nem létezik a gráfban H-út se.
\end{framed}
\begin{leftbar}
Indirekt t.f.h. van a gráfban H-kör, ez legyen $(v_1, v_2,..., v_n)$ és legyen $(v_{i_1}, v_{i_2},...,v_{i_k})$ az a k pont, amelyet elhagyva a gráf több mint k komponensre esik. Az elhagyott pontok közötti "ívek" biztosan összefüggő komponenseket alkotnak. Pl. a $(v_{i_{1}+1}, v_{i_{1}+2},..., v_{i_{2}-1})$ is összefüggő lesz, hiszen két szomszédos pontja között az eredeti H-kör éle fut. Mivel épp k ilyen ívet kapunk, ezért nem lehet több komponens k-nál (kevesebb lehet, mivel különb. ívek közt futhatnak élek). U.a. bizonyítjuk útra. Ha egy H-útból elhagyunk k pontot, legfeljebb k+1 öf. ív marad.
\end{leftbar}
\begin{framed}
ELÉGSÉGES FELTÉTEL - ORE TÉTEL Tétel: Ha az n pontú G gráfban minden olyan $x,y\in V(G)$ pontpárra, amelyre ${x,y}\in E(G)$ teljesül az is, hogy $d(x) + d(y) \geq n$, akkor a gráfban van H-kör. 
\end{framed}
\begin{leftbar}
Indirekt t.f.h. a gráf kielégíti a feltételt de nincs benne H-kör. Vegyünk hozzá a gráfhoz éleket úgy, hogy továbbra se legyen benne H-kör. Ezt egészen addig csináljuk, amíg már akárhogyan is veszünk hozzá egy élet, lesz a gráfban H-kör. Az így kapott G' gráfra továbbra is teljesül a feltétel, hiszen új élek behúzásával "rossz pontpárt" nem lehet létrehozni. Biztosan van két olyan pont, hogy ${x,y} \not \in E(G')$. Ennek a behúzásával már lesz $G' + {x,y}$-ban H-kör, tehát G'-ben van H-út. Legyen ez $P = {z_1, z_2,...,z_n}$ ahol $z_1 = x$ és $z_n = y$.
Ha x szomszédos a P út valamely $z_k$ pontjával, akkor y nem lehet összekötve $z_{k-1}$-el, mert akkor az egy H-kört adna. Így tehát y nem lehet összekötve legalább d(x) ponttal, ezért
$$d(y) \leq n - 1 - d(x)$$
ami viszont ellentmondás, hisz ${x,y} \not\in E(G)$.
\end{leftbar}
\begin{framed}
ELÉGSÉGES FELTÉTEL - DIRAC TÉTEL Tétel: Ha egy n pontú G gráfban minden pont foka legalább n/2, akkor a gráfban létezik H-kör.
\end{framed}
\begin{leftbar}
Ez az előző tételből következik, hiszen ha minden pont foka legalább n/2, akkor teljesül az Ore-tétel feltétele, mivel bármely pontpárra $$d(x) + d(y) \geq n$$
\end{leftbar}
\end{document}