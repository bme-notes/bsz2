\section{4. tétel}

\begin{definicio}{
SÍKBARAJZOLHATÓSÁG Definíció}: G \textbf{síkbarajzolható} gráf, ha lerajzolható úgy (a síkba), hogy az élei a csúcsokon kívül sehol máshol ne keresztezzék egymást.
\end{definicio}
\begin{definicio}{
TARTOMÁNYOK Definíció}: G síkbarajzolt gráf \textbf{tartományain} azon síkrészeket értjük, melyeket közrefognak az élek. Csak síkbarajzolt gráfok esetén beszélhetünk ezekről!
\end{definicio}
\begin{tetel}{
GÖMBRE RAJZOLHATÓSÁG Tétel}: G gráf pontosan akkor síkbrajzolható, ha gömbre rajzolható.
\end{tetel}
\begin{leftbar}
Bizonyítás: Egy síkban lévő gráf leképezhető gömbfelületre oly módon, hogy ezt a gömbfelületet valamelyik pontjával a síkra helyezzük, ezt a pontot tekintjük a déli pólusként, és az északi pólusból egyeneseket húzunk a gráf pontjaiba. Ezeknek a vonalaknak van metszéspontja a gömbön, ezek szolgáltatják a kívánt vetítést. Ez az ú.n. sztereografikus projekció. Ezt visszafele is meg lehet ismételni.
\end{leftbar}
\begin{tetel}{
EULER-FORMULA Tétel}: Ha egy összefüggő síkbeli gráfnak n csúcsa, e éle és t tartománya van (beleértve a külső tartományt is), akkor eleget tesz az Euler-formulának:
$$n + t = e + 2$$
\end{tetel}
\begin{leftbar}
Bizonyítás: Tekintsük a gráf egy C körét (ha van) és ennek egy $a$ élét. A C kör a síkot két részre osztja. Ezeket egyéb élek további tartományokra oszthatják, de mindkét részben van egy olyan tartomány, melynek $a$ a határa. Ha a-t elhagyjuk, a két tartomány egyesül, azaz a tartományok száma eggyel csökken. A csúcsok száma nem változik, tehát $a$ elhagyásával az $n - e + t$ érték nem változik. Ezt az eljárást addig folytassuk, amíg a gráfban nem marad kör. Ekkor viszont már csak egy feszítőfa maradt. Elég az állítást erre belátni, ami triviális, hiszen $t = 1$ és $e = n - 1$.
\end{leftbar}
\begin{tetel}{
BECSLÉS AZ ÉLEK SZÁMÁRA Tétel}: Ha G egyszerű, síkbarajzolható gráf és pontjainak a száma legalább 3, akkor az előbbi jelölésekkel:
$$e \leq 3n - 6$$
\end{tetel}
\begin{leftbar}
Bizonyítás: Vegyük G tetsz. síkbarajzolását és jelöljük az egyes tartományokat határoló élek számát $c_1, c_2...c_t$-vel. Mivel a gráf egyszerű, ezért minden tartományát legalább 3 él határolja, tehát $c_i \geq 3$. Nyilvánvaló, hogy egy élhez legfeljebb 2 tartomány tartozik, tehát ha összegezzük a tartományokat határoló élek számát minden tartományra, akkor legfeljebb $2e$-t kaphatunk. Tehát:
$$3t \leq c_1 + c_2 + ... + c_t = \sum_{i=1}^{t} c_i \leq 2e$$
Az Euler-formulát felhasználva:
$$3(e - n + 2) \leq 2e$$
Ebből átrendezéssel megkapjuk az eredményt.
\end{leftbar}
\begin{tetel}{
BECSLÉS AZ ÉLEK SZÁMÁRA Tétel}: Ha G egyszerű, síkbarajzolható gráf és minden köre legalább 4 hosszú, valamint legalább 4 pontja van, akkor:
$$e \leq 2n - 4$$
\end{tetel}
\begin{leftbar}
Minden tartományt legalább 4 él határol. Az előző biz. gondolatmenete alapján $4t \leq 2e$ és ez alapján megkapjuk a képletet.
\end{leftbar}
\begin{tetel}{
BECSLÉS MINIMÁLIS FOKSZÁMRA Tétel}: Ha G egyszerű, síkbarajzolható gráf, akkor $$\delta = min\, d(v) \leq 5$$ azaz a minimális fokszám legfeljebb 5.
\end{tetel}
\begin{leftbar}
Bizonyítás: Feltehetjük, hogy a gráf pontjainak a száma legalább 3. T.f.h. $\delta \geq 6$. Mivel a fokszámok összege egyenlő az élszámok kétszeresével, $6n \leq 2e$. Az élszám-becslés alapján azonban $2e \leq 6n - 12$, ezzel ellentmondásra jutottunk, mivel $6n \not\leq 6n - 12$.
\end{leftbar}
