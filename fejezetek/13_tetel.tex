\section{13. tétel: Floyd algoritmusa}

\url{http://cs.bme.hu/bsz2/dfs.pdf}

\begin{tetel}{FLOYD-ALGORITMUS}
  A Floyd-algoritmus a gráfban lévő összes pontpár közt megadja a távolságokat.  A sikeres futás feltétele az, hogy a gráfban NE legyen negatív összsúlyú kör.\\
  \\
  \textbf{Az algoritmus:}
  \begin{itemize}
  \item{\textbf{0.}} Minden i,j rendezett párra legyen $d^{(1)}(i,j) = l(i,j)$ és $k=2$.
  \item{\textbf{1.}} Minden i,j rendezett párra
    $$d^{(k+1)}(i,j) = min\{d^{(k)}(i,j),\,\, d^{(k)}(i,k) + d^{(k-1)}(k,j)\}$$
  \item{\textbf{2.}} Ha $k = n + 1$, akkor STOP. Különben $k = k + 1$ és folytassuk \textbf{1.} lépésnél.
  \end{itemize}

  Tehát nagyjából annyit teszünk, hogy $k = 2 ... n+1$ alkalommal javítunk egyet az egyes csúcsokhoz rendelt távolságokon a Dijkstra vagy Ford algoritmust taglaló tételhez hasonló módszerrel. Lényegi különbség, hogy itt azonban nem 
  
  Ezt a Ford-algoritmussal is megtehettük volna, viszont annak a futási ideje az összes pontból kiindítva $c\cdot ev^2$-tel lett volna arányos. A Floyd-algoritmus ezt megteszi mindössze $c\cdot v^3$ alatt.
\end{tetel}
\begin{bizonyitas}{}
T.f.h. G irányított gráf a $V(G) = {v_1, v_2,..., v_n}$ pontokon. A $v_i$-ből $v_j$-be mutató él hosszát, azaz súlyát, jelöljük l(i,j)-vel és t.f.h. a gráfban nincs negatív összsúlyú irányított kör. Ha nincs él $v_i$-ből $v_j$-be, akkor legyen $l(i,j) = \infty$. Továbbá $l(i,i) = 0$, minden $i = 1, 2,..., n$-re. Jelölje $d^{(k)}(i,j)$ a $v_i$-ből $v_j$-be vezető legrövidebb olyan irányított út hosszát, mely csak k-nál szigorúan kisebb pontokon megy át. Így $d^{(1)}(i,j) = l(i,j)$ és $d^{(n+1)}$ lesz az eredetileg keresett legrövidebb irányított út hossza lesz $v_i$-ből $v_j$-be. Világos, hogy a $v_i$-ből $v_j$-be vezető legrövidebb olyan út, ami csak $k + 1$-nél szigorúan kisebb pontokon megy át, vagy átmegy $v_k$-n, vagy nem. Ha nem megy át, akkor $d^{(k+1)}(i,j) = d^{(k)}(i,j)$. Ha viszont átmegy, akkor $d^{(k+1)}(i,j) = d^{(k)}(i,k) + d^{(k)}(k,j)$. Csak azt kell megnéznünk, mely esetben találunk rövidebb utat. Ezek után már világos, hogy az algoritmus lépésszáma $c\cdot v^3$-bel arányos.
\end{bizonyitas}


\begin{definicio}{IRÁNYÍTOTT ACIKLIKUS GRÁF}
Egy G gráfot akkor nevezünk \textbf{irányított aciclikus gráfnak} (DAG), ha irányított élei vannak és nem tartalmaz kört.
\end{definicio}

\begin{definicio}{TOPOLOGIKUS ELRENDEZÉS LÉTEZÉSE}
Legyen G egy irányított gráf. G topologikus elrendezése a csúcsoknak egy olyan $v_1, v_2,..., v_n$ sorrendje, melyben $x\rightarrow y \in E$ esetén x előbb van, mint y (azaz ha $x = v_i, y = v_j,\,\,akkor\,\,i<j$)
\end{definicio}

\begin{tetel}{TOPOLOGIKUS ELRENDEZÉS Lemma}
Ha G irányított gráf aciklikus, akkor létezik benne nyelő (olyan pont, amiből nincsen kimenő él).
\end{tetel}

\begin{bizonyitas}{}
Legyen P a leghosszabb irányított utak egyike, v legyen a végpontja. T.f.h. v nem nyelő. Járjuk be az utat topologikus sorrendben. Az előző állítás annyit jelent, hogy P vagy nem a legutolsó elem a topologikus elrendezésben (ellentmondás) vagy egy, a topologikus sorrendben előrébb lévő ponthoz csatlakozik vissza (ellentmondás).
\end{bizonyitas}

\begin{tetel}{TOPOLOGIKUS ELRENDEZÉS}
Egy G irányított gráfhoz akkor és csak akkor létezik topologikus elrendezés, ha az aciklikus.
\end{tetel}

\begin{bizonyitas}{}
A szükségeset triviális bizonyítani, viszont az elégségességet nem. Az előbbi lemma állítását felhasználjuk a bizonyításhoz. Keressünk ebben a gráfban egy nyelőcsúcsot, ez a $v_n$ csúcs. Ezt dobjuk ki. Ekkor $G\backslash{v_n}$-ben $v_{n-1}$ lesz nyelő. Ezt is dobjuk ki. Ismételjük, amíg semmi se marad. Amit ``kidobtunk'', ha fordítva sorba rendezzük (tehát a legvégén lesz az először kidobott elem), akkor az egy topologikus sorbarendezése lesz G-nek. A DFS is generál egy ilyet a futása során, ha nincs benne visszaél (tehát ha nincs benne kör).
\end{bizonyitas}

\begin{tetel}{LEGRÖVIDEBB ÉS LEGHOSSZABB ÚT KERESÉSE DAG-BAN Algoritmus}
A topologikus rendezést használva lineáris időben, tehát $n+e$-vel arányos lépésszámban megoldható a leghosszabb/legrövidebb út keresése.
Input legyen a G irányított gráf és annak egy topologikus sorrendje (s).
\\
\textbf{Az algoritmus:}\\
Legrövidebb út kereséséhez a korábbiakhoz hasonlóan minden csúcshoz végtelen nagy távolságot rendelünk hozzá, a kezdő csúcshoz pedig egyet.

Utána topologikus sorrendben haladunk a csúcsokon, és a kimenő élek végpontjain lévő csúcsoknál megpróbáljuk javítani a korábbiakhoz szintén hasonlóan a távolságot: ha a mi irányunkból összeségében kisebb távolság jön ki a csúcsra, kicseréljük, egyébként hagyjuk, ami eleve ott volt. Ha egy adott pontból így minden kimenő éllel végeztünk, haladunk tehát a topologikus sorrendben a következő pontra.

Két pont között az út hasonlóan található meg: az összes él hosszát szorozzuk meg -1-gyel és keressük meg ismét a legrövidebb utat (tehát abszolút értékben ezúttal a legnagyobb számot). A leghosszabb utak az egyes csúcsokba így a csúcskhoz rendelt értékek abszolút értékei lesznek.
\end{tetel}
